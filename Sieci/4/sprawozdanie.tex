\documentclass{article}
%packages
% \usepackage{tocloft}
\usepackage{polski}
\usepackage{amsmath}
\usepackage[utf8]{inputenc}
\usepackage{graphicx}
\usepackage{indentfirst}
\usepackage{float}
\usepackage[font=small,labelfont=bf]{caption}
\usepackage[polish]{babel}
\usepackage{hyperref}
\hypersetup{
    colorlinks,
    citecolor=black,
    filecolor=black,
    linkcolor=black,
    urlcolor=black
}
% Variables
\newcommand{\warningSymbol}{\includegraphics[height=10px]{graph/warning2.png}}
\newcommand{\HRule}{\rule{\linewidth}{0.5mm}}
\newcommand{\Prowadzacy}{dr hab. inż. Robert \textsc{Nowicki} prof. PCz}
\newcommand{\Ja}{Piotr \textsc{Filek}\\101311\\I grupa}
\newcommand{\DataLaboratorium}{29 października 2013}
\newcommand{\Uczelnia}{ \textsc{\LARGE Politechnika Częstochowska}\\[1.5cm]}
\newcommand{\Przedmiot}{ \textsc{\Large Podstawy Sieci Komputerowych}\\[1.5cm]}
\newcommand{\TytulLaboratirum}{Laboratorium 4\\Testery okablowania}
\frenchspacing


\begin{document}
\begin{titlepage}
\begin{center}
\Uczelnia
% \textsc{\LARGE Politechnika Częstochowska}\\[1.5cm]
\Przedmiot% \textsc{\Large Final year project}\\[0.5cm]
\HRule\\[0.4cm]
{ \huge \bfseries \TytulLaboratirum \\[0.4cm] }
% { \huge \bfseries Large brewing techniques \\[0.4cm]}
\HRule\\[1.5cm]

% Author and supervisor
\begin{minipage}{0.4\textwidth}
\begin{flushleft} \large
\emph{Autor:}\\
\Ja
\end{flushleft}
\end{minipage}
\begin{minipage}{0.4\textwidth}
\begin{flushright} \large
\emph{Prowadzący:} \\
\Prowadzacy
\end{flushright}
\end{minipage}

\vfill

% Bottom of the page
{\large \today}

\end{center}
\end{titlepage}
\newpage
\section{Cel laboratorium}
Celem laboratorium było zapoznanie się z działaniem testerów okablowania sieci LAN i zbadanie z ich użyciem przykładowych przewodów stacyjnych.
\section{Wyniki}
\subsection{Fluke Networks LinkRunner}
\begin{enumerate}
% 01 %
\item 01 - 1 metr - problem z wtyczką - OK
% 03 %
\item 03 - 3 metry - scrossowany
% 04 %
\item 04 - 3 metry
  \begin{itemize}
  \item 3 6 - \warningSymbol
  \item 4 5 - \warningSymbol
  \end{itemize}
  % 05 %
\item 05 - 3 metry \warningSymbol
  % 06 %
\item 06 - 3 metry - problem z wtyczką
  % 07 %
\item 07 - 3 metry
  \begin{itemize}
  \item 3 6 - \warningSymbol
  \item 4 5 - \warningSymbol
  \item 7 8 - przerwany
  \end{itemize}
  % 08 %
\item 08 - \warningSymbol
  \begin{itemize}
  \item 1 2 - 5 metrów
  \item 3 6 - 2 metry
  \item 4 5 - 2 metry
  \item 7 8 - 5 metrów
  \end{itemize}
% 09 %
\item 09 - 4 metry
  \begin{itemize}
  \item 1 2 - \warningSymbol
  \item 3 6 - \warningSymbol
  \item 4 5 - OK
  \item 7 8 - przerwany 8
  \end{itemize}

\item Nieoznaczony - 2 metry - OK
\item Nieoznaczony - 2 metry - uszkodzony fizycznie
  \begin{itemize}
  \item 3 6 - przerwany
  \item 4 5 - przerwany
  \end{itemize}
\end{enumerate}
\subsection{CHL 468}
\begin{minipage}{0.4\textwidth}
  \begin{enumerate}
  \item 01 - OK
  \item 03 - scrossowany 7/8
  \item 04
    \begin{itemize}
    \item 1 nie łączy
    \item 3 z 5
    \item 4 z 6
    \item 5 z 2
    \item 6 z 4
    \item 7 z 8
    \end{itemize}
  \item 05
    \begin{itemize}
    \item 1 z 3
    \item 2 z 6
    \item 3 z 1
  \item 4 z 8
  \item 5 z 7
  \item 6 z 2
  \item 7 z 5
  \item 8 z 4
  \end{itemize}
\item 06 - OK
\item 07
  \begin{itemize}
  \item 1 z 1
  \item 2 z 2
  \item 3 z 5
  \item 4 z 4
  \item 5 z 3
  \item 6 z 6
  \item 7 z 7
  \item 8 z 8
  \end{itemize}
\item 08 - OK
\end{enumerate}
\end{minipage}
\begin{minipage}{0.4\textwidth}
  \begin{enumerate}
    \setcounter{enumi}{7}    
    
  \item 09
    \begin{itemize}
    \item 1 z 3
  \item 2 z 6
  \item 3 z 1
  \item 4 z 4
  \item 5 z 5
  \item 6 z 2
  \item 7 z 7
  \item 8 z 8
  \end{itemize}
\item Nieoznaczony
  \begin{itemize}
  \item 1 z 3
  \item 2 z 6
  \item 3 z 1
  \item 4 z 4
  \item 5 z 5
  \item 6 z 2
  \item 7 z 7
  \item 8 z 8
  \end{itemize}
\item Nieoznaczony czarny - OK
\item Nieoznaczony uszkodzony fizycznie
  \begin{itemize}
  \item 1 z 1
  \item 2 z 2
  \item 3 z 3
  \item 4 - przerwany
  \item 5 - przerwany
  \item 6 - przerwany
  \item 7 z 7
  \item 8 z 8
  \end{itemize}

\end{enumerate}
\end{minipage}
\newpage
\subsection{Fluke Networks MicroScanner Pro}
\begin{enumerate}
\item 01 - OK
\item 03
  \begin{itemize}
  \item 1 2 3 4 5 6 7 8
  \item 1 2 3 4 5 6 8 7
  \end{itemize}
\item 04
  \begin{itemize}
  \item 1 2 3 4 5 6 7 8
  \item 1 2 5 6 3 4 7 8
  \end{itemize}
\item 05
  \begin{itemize}
  \item 1 2 3 4 5 6 7 8
  \item 3 6 1 8 7 2 5 4
  \end{itemize}
\item 06 - OK
\item 07
  \begin{itemize}
  \item 1 2 3 4 5 6 7 8
  \item 1 2 x x x x 7 8
  \end{itemize}
\item 08 - Split par - \textbf{pogrubione cyfry migały}
  \begin{itemize}
  \item 1 2 3 4 5 6 7 8
  \item 1 2 \textbf{3 4 5 6} 7 8 
  \end{itemize}
\item 09
  \begin{itemize}
  \item 1 2 3 4 5 6 7 8
  \item 3 6 1 4 5 2 7 8
  \end{itemize}
\item Nieoznaczony
  \begin{itemize}
  \item 1 2 3 4 5 6 7 8
  \item 3 6 1 4 5 2 7 8
  \end{itemize}
\item Nieoznaczony czarny - OK
\item Nieoznaczony - uszkodzony fizycznie
  \begin{itemize}
  \item 1 2 x x x x 7 8
  \item 1 2 x x x x 7 8
  \end{itemize}
\end{enumerate}
\end{document}