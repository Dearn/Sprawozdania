\documentclass{article}
% packages
% \usepackage{tocloft}
\usepackage{polski}
\usepackage{amsmath}
\usepackage[utf8]{inputenc}
\usepackage{graphicx}
\usepackage{indentfirst}
\usepackage{float}
\usepackage[font=small,labelfont=bf]{caption}
\usepackage[polish]{babel}
\usepackage{hyperref}
\hypersetup{
  colorlinks,
  citecolor=black,
  filecolor=black,
  linkcolor=black,
  urlcolor=black
}
% Variables
\newcommand{\HRule}{\rule{\linewidth}{0.5mm}}
\newcommand{\Prowadzacy}{dr inż. Krzysztof \textsc{Rojek}}
\newcommand{\Ja}{Piotr \textsc{Filek}\\101311\\I grupa}
\newcommand{\DataLaboratorium}{1 listopada 2013}
\newcommand{\Uczelnia}{ \textsc{\LARGE Politechnika Częstochowska}\\[1.5cm]}
\newcommand{\Przedmiot}{ \textsc{\Large Grafika Komputerowa i Wizualizacja}\\[1.5cm]}
\newcommand{\TytulLaboratirum}{Laboratorium 4\\OpenGL}
\frenchspacing

% \setlength{\intextsep}{20pt plus 1.0pt minus 2.0pt}

% Equations list
% \newcommand{\listequationsname}{List of Equations}
% \newlistof{myequations}{equ}{\listequationsname}
% \newcommand{\myequations}[1]{%
% \addcontentsline{equ}{myequations}{\protect\numberline{\theequation}#1}\par}

\begin{document}
\begin{titlepage}
\begin{center}
\Uczelnia
% \textsc{\LARGE Politechnika Częstochowska}\\[1.5cm]
\Przedmiot% \textsc{\Large Final year project}\\[0.5cm]
\HRule\\[0.4cm]
{ \huge \bfseries \TytulLaboratirum \\[0.4cm] }
% { \huge \bfseries Large brewing techniques \\[0.4cm]}
\HRule\\[1.5cm]

% Author and supervisor
\begin{minipage}[t]{0.4\textwidth}
\begin{flushleft}\large
\emph{Autor:}\\
\Ja
\end{flushleft}
\end{minipage}
\begin{minipage}[t]{0.5\textwidth}
\begin{flushright} \large
\emph{Prowadzący:} \\
\Prowadzacy
\end{flushright}
\end{minipage}

\vfill

% Bottom of the page
{\large \DataLaboratorium}

\end{center}
\end{titlepage}
% \begin{tableofcontents}
%   \listoffigures
% \end{tableofcontents}
\newpage
\section{Cel laboratorium}
Celem laboratorium było zapoznanie się z obsługą programu \emph{GNU Make} oraz z podstawowymi instrukcjami \newline biblioteki \emph{OpenGL}. Program został napisany w języku \emph{C++} wykorzystując biblioteki \emph{QT} oraz \emph{OpenGL}.
\section{Przebieg laboratorium}
Podczas laboratorium zadaniem było stworzenie sceny, która przedstawiała:
\begin{itemize}
\item sześcian o wymiarach 2x2x2, o różnokolorowych ścianach
\item linie w płaszczyźnie XZ przechodzącą przez jego środek
\item obrócenie sześcianu pod kątem 45 stopni względem osi X, Y i Z
\item podłogę wykonaną z trójkątów (każdy wierzchołek z innym kolorem)
\end{itemize}
Wykonując powyższe zadania, zapoznaliśmy się z takimi instrukcjami jak:
\begin{itemize}
\item \textsf{glTranslatef} - funkcja służąca do zmiany pozycji na której rysuje się obiekt. Kolejne przesunięcia odnoszą się w stosunku do siebie, a nie pozycji początkowej
\item używając \textsf{glBegin} mogliśmy tworzyć takie figury jak trójkąt (\textsf{GL\_TRIANGLES}), kwadrat (\textsf{GL\_QUADS}), czy linia (\textsf{GL\_LINES})
\item \textsf{glColor3f} do przypisania koloru wierzchołków (takie same - jednolity kolor, różne - gradient)
\item \textsf{glRotatef} do rotacji obiektów (przechylenie sześcianu) oraz jego obrotu (animacja)
\item \textsf{glLoadIdentity} resetuje pozycje oraz nachylenie do wartości domyślnej


\end{itemize}
\vfill
Kod programu dostępny pod adresem: \newline \url{http://github.com/Dearn/Grafika/tree/master/1}
\end{document}