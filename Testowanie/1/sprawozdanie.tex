\documentclass{article}
% packages
% \usepackage{tocloft}
\usepackage{polski}
\usepackage{amsmath}
\usepackage[utf8]{inputenc}
\usepackage{graphicx}
\usepackage{ucs}
\usepackage{indentfirst}
\usepackage{float}
\usepackage[font=small,labelfont=bf]{caption}
\usepackage[polish]{babel}
\usepackage{listings}
\usepackage{hyperref}
\usepackage{color}

\definecolor{mygreen}{rgb}{0,0.6,0}
\definecolor{mygray}{rgb}{0.5,0.5,0.5}
\definecolor{mymauve}{rgb}{0.58,0,0.82}

\hypersetup{
  colorlinks,
  citecolor=black,
  filecolor=black,
  linkcolor=black,
  urlcolor=black
}
% Variables
\newcommand{\HRule}{\rule{\linewidth}{0.5mm}}
\newcommand{\Prowadzacy}{dr inż. Grzegorz \textsc{Michalski}}
\newcommand{\Ja}{Piotr \textsc{Filek}\\Marcin \textsc{Nowak}\\II grupa IO}
\newcommand{\DataLaboratorium}{9 października 2014}
\newcommand{\Uczelnia}{ \textsc{\LARGE Politechnika Częstochowska}\\[1.5cm]}
\newcommand{\Przedmiot}{ \textsc{\Large Testowanie Oprogramowania}\\[1.5cm]}
\newcommand{\TytulLaboratirum}{Laboratorium 1}
%% \frenchspacing

\lstset{
  language=SQL,
  inputencoding=utf8x, 
  extendedchars=\true,
  literate={ą}{{\k{a}}}1
  {Ą}{{\k{A}}}1
  {ę}{{\k{e}}}1
  {Ę}{{\k{E}}}1
  {ó}{{\'o}}1
  {Ó}{{\'O}}1
  {ś}{{\'s}}1
  {Ś}{{\'S}}1
  {ł}{{\l{}}}1
  {Ł}{{\L{}}}1
  {ż}{{\.z}}1
  {Ż}{{\.Z}}1
  {ź}{{\'z}}1
  {Ź}{{\'Z}}1
  {ć}{{\'c}}1
  {Ć}{{\'C}}1
  {ń}{{\'n}}1
  {Ń}{{\'N}}1
}

% \setlength{\intextsep}{20pt plus 1.0pt minus 2.0pt}

% Equations list
% \newcommand{\listequationsname}{List of Equations}
% \newlistof{myequations}{equ}{\listequationsname}
% \newcommand{\myequations}[1]{%
% \addcontentsline{equ}{myequations}{\protect\numberline{\theequation}#1}\par}

\begin{document}
\begin{titlepage}
\begin{center}
\Uczelnia
% \textsc{\LARGE Politechnika Częstochowska}\\[1.5cm]
\Przedmiot% \textsc{\Large Final year project}\\[0.5cm]
\HRule\\[0.4cm]
{ \huge \bfseries \TytulLaboratirum \\[0.4cm] }
% { \huge \bfseries Large brewing techniques \\[0.4cm]}
\HRule\\[1.5cm]

% Author and supervisor
\begin{minipage}[t]{0.4\textwidth}
\begin{flushleft}\large
\emph{Autor:}\\
\Ja
\end{flushleft}
\end{minipage}
\begin{minipage}[t]{0.5\textwidth}
\begin{flushright} \large
\emph{Prowadzący:} \\
\Prowadzacy
\end{flushright}
\end{minipage}

\vfill

% Bottom of the page
{\large \DataLaboratorium}

\end{center}
\end{titlepage}
% \begin{tableofcontents}
%   \listoffigures
% \end{tableofcontents}
\newpage
\section{Cel laboratorium}
Celem laboratorium było napisanie prostego kalkulatora w wybranym języku (C++ w naszym przypadku), oraz wykonanie testów do kalkulatora napisanego przez inną osobę (Damian Łukasik).

\section{Przebieg laboratorium}
\subsection{Działanie programu}

\begin{lstlisting}
  Witam w Kalkulatorze
  Wybierz:
  0-wyjście,
  1-dodawanie,
  2-odejmowanie,
  3-mnozenie,
  4-dzielenie,
  5-pierwiastek,
  6-równanie kwadratowe
  Wyieram=

\end{lstlisting}

\subsection{Przypadki testowe}
\subsubsection{Testy zakończone sukcesem}
\begin{enumerate}
\item Wybranie opcji 0-6 - zostajemy poproszeni o podanie liczb
\item Wybranie opcji spoza zakresu - ponowne poproszenie o podanie liczby 0-6
\item Wpisanie litery zamiast cyfry - zakończenie programu
\item Dodawanie dwóch liczb całkowitych (1, 2) - poprawny wynik
\item Dodawanie dwóch liczb rzeczywistych (1.23, -0.12) - poprawny wynik
\item Odejomowanie dwóch liczb całkowitych (1, 2) - poprawny wynik
\item Odejomowanie dwóch liczb rzeczywistych (1.23, -0.12 - poprawny wynik
\item Mnożenie dwóch liczb całkowitych (1, 2) - poprawny wynik
\item Mnożenie dwóch liczb rzeczywistych (1.23, -0.12 - poprawny wynik
\item Dzielenie dwóch liczb całkowitych (1, 2) - poprawny wynik
\item Dzielenie dwóch liczb rzeczywistych (1.23, -0.12 - poprawny wynik
\item Dzielenie przez zero (25, 0) - komunikat "Nie dzieli się przez zero!!!"
\item Pierwiastkowanie liczby naturalnej 5 - poprawny wynik
\item Sprawdzenie, czy równanie kwadratowe wykrywa zero miejsc zerowych dla wartości (1, 2, 5)
\item Sprawdzenie, czy równanie kwadratowe wykrywa jedno miejsce zerowe dla wartości (5, 5, 1.25)
\end{enumerate}
\subsubsection{Testy zakończone niepowodzeniem}
\begin{enumerate}

\item Podanie nieprawidłowego znaku  zamiast liczby przy jakimkolwiek działaniu - zapętlenie się programu
\item Sprawdzenie, czy równanie kwadratowe zwraca błąd gdy a=0 przy podanym wejściu (0, 5, 1.25) - wykrywa dwa miejsca zerowe -inf oraz -nan  

\end{enumerate}




\lstset{ %
  backgroundcolor=\color{white},   % choose the background color; you must add \usepackage{color} or \usepackage{xcolor}
  basicstyle=\footnotesize,        % the size of the fonts that are used for the code
  breakatwhitespace=false,         % sets if automatic breaks should only happen at whitespace
  breaklines=true,                 % sets automatic line breaking
  captionpos=b,                    % sets the caption-position to bottom
  commentstyle=\color{mygreen},    % comment style
  deletekeywords={...},            % if you want to delete keywords from the given language
  escapeinside={\%*}{*)},          % if you want to add LaTeX within your code
  extendedchars=true,              % lets you use non-ASCII characters; for 8-bits encodings only, does not work with UTF-8
  frame=single,                    % adds a frame around the code
  keepspaces=true,                 % keeps spaces in text, useful for keeping indentation of code (possibly needs columns=flexible)
  keywordstyle=\color{blue},       % keyword style
  language=Scilab,                 % the language of the code
  morekeywords={*,...},            % if you want to add more keywords to the set
  numbers=left,                    % where to put the line-numbers; possible values are (none, left, right)
  numbersep=5pt,                   % how far the line-numbers are from the code
  numberstyle=\tiny\color{mygray}, % the style that is used for the line-numbers
  rulecolor=\color{black},         % if not set, the frame-color may be changed on line-breaks within not-black text (e.g. comments (green here))
  showspaces=false,                % show spaces everywhere adding particular underscores; it overrides 'showstringspaces'
  showstringspaces=false,          % underline spaces within strings only
  showtabs=false,                  % show tabs within strings adding particular underscores
  stepnumber=1,                    % the step between two line-numbers. If it's 1, each line will be numbered
  stringstyle=\color{mymauve},     % string literal style
  tabsize=2,                       % sets default tabsize to 2 spaces
  title=\lstname                   % show the filename of files included with \lstinputlisting; also try caption instead of title
}
\subsection{Kod Programu}
Autor: Damian Łukasik
\lstinputlisting[title=lab1ProgramDamiana.cpp]{lab1ProgramDamiana.cpp}

\end{document}
